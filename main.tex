\documentclass{article}
\usepackage[utf8]{inputenc}

\title{Heisenberg's Uncertainty Principle}
\author{Yatharath}
\date{July 2021}

\begin{document}

\maketitle

\section*{About The Principle}
\paragraph{Introduced by Werner Heisenberg in 1927, the uncertainty principle is a limit on quantum mechanics. It states that the more certain you are about a particle's momentum (P) the less certain you are about the particle's position (x) ie. momentum and position can never both be known exactly. A common misconception is that this effect is due to a problem with the measuring procedure. This is incorrect, it is a limit on accuracy fundamental to quantum mechanics. The right hand side involves Plank's constant (h) which is equal to a tiny value (a decimal with 33 zeros), which is why this effect isn't observed in our everyday, "classical", experience.}
\paragraph{This is to make sure this compiles}
\section*{The Mathematical Equation}
\begin{center}
\section*{${ \Delta }{ x }{ \Delta }{\rho }\quad\ge\quad\frac{ \hbar }{ 2 }$}
\end{center}

\end{document}
